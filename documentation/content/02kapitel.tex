%!TEX root = ../dokumentation.tex

\chapter{Zielsetzung}

Ziel des Projektes ist die Realisierung einer plattformunabhängigen App.
Dies geschieht mit dem Framework Flutter und der dazugehörigen Programmiersprache Dart.
Die App soll dabei einen Bezahlvorgang durch das Scannen eines QR-Codes ermöglichen.
Unsere App trägt den Namen \glqq Payero\grqq{} und läuft auf iOS und Android.
Die App bietet folgende Funktionen:

\begin{itemize}
    \item Scannen eines QR-Codes
    \begin{itemize}
        \item Nur Payero eigene QR-Codes zulassen
        \item BenutzerID im QR-Code hinterlegen
        \item Betrag im QR-Code hinterlegen
    \end{itemize}
    \item Anbieten mehrerer Bezahlverfahren
    \begin{itemize}
        \item PayPal
        \item Kreditkarten
        \begin{itemize}
            \item VISA
            \item Mastercard
            \item AMEX (über GooglePay und ApplePay)
        \end{itemize}
        \item GooglePay
        \item ApplePay (nach Eingabe einer gültigen Merchant-ID)
    \end{itemize}
    \item Datenbankanbindung
    \begin{itemize}
        \item Speichert BenutzerID
        \item Speichert Transaktionen eines Benutzers
        \item Erzeugt dynamische nutzerspezifische QR-Codes
    \end{itemize}
    \item Benutzerkonto
    \begin{itemize}
        \item Enthält Transaktionshistorie
        \item Enthält eigenen QR-Code
    \end{itemize}
    \item Einfache Bedienbarkeit durch Benutzer
    \item Designlanguage mit einer eigenen Corporate Identity mit MaterialDesign
\end{itemize}