%!TEX root = ../dokumentation.tex

\chapter{Testing}

Umfangreiche Tests wurden durchgeführt, um sicherzustellen, dass die Anwendung auf den verschiedenen Plattformen wie erwartet funktioniert.
Die Tests wurden auf verschiedenen Emulatoren und Endgeräten durchgeführt.
Die folgenden Tests wurden durchgeführt:

\begin{itemize}
    \item Regressionstests
    \item Funktionsprüfung
    \item Integrationstests
    \item Systemtests
    \item Leistungsprüfung
    \item GUI-Tests
    \item Abnahmeprüfung
\end{itemize}

Während der Entwicklung wurden die oben genannten Tests wiederholt durchgeführt.
Durch Funktions- und Integrationstests wurde immer wieder sichergestellt, dass das Implementierte auch funktioniert.
Dazu wurden immer wieder die Emulatoren und auch die realen Endgeräte eingesetzt.
Durch die verschiedenen Softwareversionen wurde die Chance Fehler zu finden erhöht.
Im Anschluss wurden immer wieder Systemtests durchgeführt, um feststellen zu können, ob weitere Funktionen nicht mehr wie geplant funktionieren.
Dabei konnten Fehler gefunden werden, wie z.B. dass PayPal unter iOS problemlos funktionierte, unter Android aber nicht.
Der Fehler konnte durch eine Änderung in der build.gradle behoben werden.
Dazu wurde die applicationId auf eine com.example... URL geändert und alle Unterstriche entfernt.
Damit funktionierte der PayPal Bezahlvorgang unter Android wieder.
Zusätzlich wurden regelmäßig GUI Tests durchgeführt.
Damit wurde sichergestellt, dass die Anzeigen und GUI Elemente auch auf unterschiedlich großen Geräten gut aussehen und bedienbar sind.
Ein einfacher Performancetest wurde ebenfalls regelmäßig durchgeführt.
Dieser diente lediglich dazu, festzustellen, ob es Probleme und lange Ladezeiten bei Funktionen oder Seiten gab.
Dazu wurden die Zeiten gemessen, um lange Wartezeiten zu erkennen.
Die oben beschriebenen Tests wurden während der Entwicklung mehrmals kontinuierlich durchgeführt.
Ganz am Ende, kurz vor der Fertigstellung und Präsentation, gab es noch einen größeren Abnahmetest.
Dabei wurden alle oben genannten Tests durchgeführt, um sicherzustellen, dass alles korrekt funktioniert und auch vom Design her gut aussieht.
Dabei wurde festgestellt, dass der Zugriff auf den Backend-Server im \glqq eduroam\grq{} WLAN blockiert wurde.
Dies verhinderte den Registrierungsprozess zu Beginn und machte die App somit unbrauchbar.
Durch den Wechsel zu einem anderen WLAN konnte das Problem behoben werden.
Somit konnte sichergestellt werden, dass die Anwendung zum Zeitpunkt der Präsentation wie erwartet funktionierte.
