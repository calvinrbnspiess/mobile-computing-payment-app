%!TEX root = ../dokumentation.tex

\chapter{Ausblick und Weiterentwicklung}

Die App kann noch beliebig weiterentwickelt und unter bestimmten Voraussetzungen auch produktiv genutzt werden.
Um das volle Potential der App nutzen zu können, bräuchte man aber definitiv einen BrainTree Product Account.
Außerdem bräuchte man noch einen Apple Developer Account, um ApplePay anbieten zu können.\\
Mit mehr Zeit wäre es auch möglich gewesen, weitere Features zu implementieren.
Dazu gehört auf jeden Fall ein guter Darkmode, der auch dem Design der Corporate Identity von \glqq Payero\grqq{} entspricht.
Wünschenswert wäre auch eine Option zur Unterstützung mehrerer Sprachen.
Neben Deutsch sollten dies auf jeden Fall Englisch, Französisch und Spanisch sein.
Dies könnte aber auch auf die 24 Amtssprachen der EU erweitert werden, um den gesamten europäischen Markt abzudecken.
Die Amtssprachen der EU sind: Bulgarisch, Dänisch, Deutsch, Englisch, Estnisch, Finnisch, Französisch, Griechisch, Irisch, Italienisch, Kroatisch, Lettisch, Litauisch, Maltesisch, Niederländisch, Polnisch, Portugiesisch, Rumänisch, Schwedisch, Slowakisch, Slowenisch, Spanisch, Tschechisch und Ungarisch.
BrainTree unterstützt offiziell 23 Sprachen, aber es ist möglich, weitere Sprachen hinzuzufügen.
Auch für neu zu erschließende Märkte kann die Anzahl der Sprachen erweitert werden.
Dabei ist allerdings zu beachten, in welchen Ländern BrainTree und PayPal jeweils operieren, da nicht alle Länder weltweit unterstützt werden.
Ein weiteres benutzerfreundliches Feature wäre die Implementierung des Scannens von Kreditkarten.
Mit diesem Feature kann man seine Kreditkarte zum Bezahlen sicher einscannen, ohne die Nummern mühsam eintippen zu müssen.
Dies erhöht den Benutzerkomfort.
Dies kann innerhalb von BrainTree realisiert werden.
Dies liegt daran, dass PayPal das ehemalige Startup card.io übernommen hat, das die Technologie dafür besaß.
Es ist auch möglich, den verwendeten Backend-Server so zu erweitern, dass er Client-Token generiert.
Diese sind pro Transaktion einmalig und ermöglichen es, Transaktionen eines Nutzers sicher nachzuvollziehen.
Dazu muss dann kein TransactionKey mehr in den Code eingebaut werden.
Vor allem bei der Implementierung des 3D Secure 2.0 Verfahrens für Kreditkarten ist dies sehr hilfreich.
Dieses Verfahren ist vor allem im produktiven Einsatz essentiell und erhöht die Akzeptanz von Kreditkarten.
Um das Design der App noch schöner zu gestalten, kann man anstelle des BrainTree Dropin UI ein eigenes Design für die jeweiligen UI Elemente erstellen.
Dies kann in BrainTree mit Hosted Fields realisiert werden, die individuell generiert und angepasst werden können.
Dieser Aufwand ermöglicht eine noch bessere Umsetzung des eigenen Designs auf Basis einer Corporate Identity.
