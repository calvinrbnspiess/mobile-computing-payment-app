%!TEX root = ../dokumentation.tex

\chapter{Softwaretechnik}
Diese Kapitel beinhaltet alle benötigten Software- und Hardwaretools, sowie die benutzen Versionen.
\section{Softwarepakete}
Windows 10 Enterprise 22H2

\section{Entwicklungsumgebung}
Visual Studio Code v.1.84.2

\section{Emulation}

Die App wurde sowohl im Simulator, als auch auf zwei physischen Endgeräten getestet.
Die verwendeten Geräte umfassen:

iOS:
-	Simulator: iPhone SE (3rd Generation), iOS 17.0
Die iOS-Simulation wurde mit dem SimulatorKit von Apple in der Version 15.0.1 (1015.2) ausgeführt.

Android:
-	Simulator: Pixel 3a, Api-Version 27, Android 8.1 (Oreo)
-	Simulator: Pixel 3a, Api-Version 34, Android 12
-	Simulator: Pixel 2, Api-Version 34, Android 12

\begin{figure}[H]
    \centering
    \includegraphics[width=0.8\textwidth]{images/android_emulator_reibenspiess.png}
    \caption{Einstellungen des Android-Emulators von Calvin Reibenspieß.}
    \label{branding}
\end{figure}

\begin{figure}[H]
  \centering
  \includegraphics[width=0.8\textwidth]{images/android_emu_fb_1.png}
  \caption{Einstellungen des Android-Emulators von Florian Brändle Pixel 3a.}
\end{figure}

\begin{figure}[H]
  \centering
  \includegraphics[width=0.8\textwidth]{images/android_emu_fb_2.png}
  \caption{Einstellungen des Android-Emulators von Florian Brändle Pixel 2.}
\end{figure}

\section{Hardware}

-	iPhone 12 Pro, iOS 16.1.2
IMEI: 35 861174 3743835

